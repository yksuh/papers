\documentclass[times]{speauth}
\usepackage{moreverb}
\usepackage{url}
\usepackage{textcomp}
\usepackage{amssymb,amsmath}
\usepackage{balance}
\usepackage[ruled,linesnumbered]{algorithm2e}
\usepackage{subfigure}
\usepackage{float}
\newtheorem{theorem}{Theorem}[section]
\newtheorem{lemma}[theorem]{Lemma}
\usepackage{latexsym}
\usepackage{multirow}
\usepackage{color}
\usepackage{url}
\usepackage{comment}
\usepackage{graphicx}           
\usepackage{epsfig}
\usepackage{algorithmic}
\usepackage{listings}
\usepackage{multirow}
\usepackage{etoolbox}
%\def\ert{\mbox{{\it expRunTime}}}
%\def\rpe{\mbox{{\it recordImageDiff}}}
%\def\procDiff{\mbox{{\it procDiff}}}
%\def\uintMax{\mbox{{\tt UINT\_MAX}}}
%\def\numRep{\mbox{{\tt NUM\_REPETITIONS}}}
%\def\DiffA{\mbox{{\it Diff$_{A}$}}}
%\def\DiffB{\mbox{{\it Diff$_{B}$}}}
%\def\getProcDiff{\mbox{{\it getProcDiff}}}
%\def\bfps{\mbox{{\it beforeImage}}}
%\def\afps{\mbox{{\it afterImage}}}
%\def\gpi{\mbox{{\it getProcInfo}}}
%\def\stm{\mbox{{\it startTime}}}
%\def\gtm{\mbox{{\it getTime}}}
%\def\prog{\mbox{{\it userProgram}}}
%\def\eput{\mbox{{\it executeProgram}}}
%\def\etm{\mbox{{\it endTime}}}
%\def\ect{\mbox{{\it elapsedClockTime}}}
%\def\et{\mbox{{\it elapsedTime}}}
%\def\pt{\mbox{{\it ProgramTime}}}
%\def\cpt{\mbox{{\it computeProgramTime}}}
%\def\inc{\mbox{INC}}
\def\azdblab{\mbox{{\sc AZDBLab}}}
\definecolor{grey}{RGB}{200,200,200}
\newcommand{\hilite}[1]{\colorbox{grey}{#1}}
\newcommand{\hilitey}[1]{\colorbox{yellow}{#1}}
\newcommand{\hiliting}[1]{\colorbox{grey}{#1}}
\long\def\todo#1{\hilitey{{\bf TODO: [}} {\em #1}\hilitey{]}}
\long\def\shorten#1{}
\makeatletter
\def\@xfootnote[#1]{
  \protected@xdef\@thefnmark{#1}
  \@footnotemark\@footnotetext}
\makeatother
%c2j: Conference to Journal: first parameter is conference, second is journal
% For Conferences: c2j#1#2{#1}
% For Journals: c2j#1#2{2}
\long\def\c2j#1#2{#1}
\def\volumeyear{2017}
\begin{document}

\runningheads{Y-K.~Suh, R.~T.~Snodgrass}{$\azdblab$: A Laboratory Information System for Large-Scale Empirical DBMS Studies}

\title{$\azdblab$: A Laboratory Information System \\ for Large-Scale Empirical DBMS Studies}

\author{Young-Kyoon Suh\affil{1}\corrauth, Richard T.~Snodgrass\affil{1}
}

\address{
\affilnum{1}University of Arizona, Tucson, AZ 85721
}

\corraddr{KISTI, 245 Daehak-ro, Yuseong-gu, Daejeon, 34141, Rep. of KOREA (current address).
E-mail: \hbox{yksuh}@kisti.re.kr}

\begin{abstract}
In database field, scientific approach has been much less prominent while very stong
mathematical and engineering work has been done.
Understanding database query optimizers is of critical importance in a sense that it can provide
great insights into their improvements by looming which parts should be reexamined.
However, there have been few systems for supporting this scientific approach in which one can
simultaneously run and monitor a variety of experiments on DBMSes and analyze the
results to test his/her hypotheses.

In this manuscript we present a novel DBMS-oriented research infrastructure, 
called {\em Arizona Database Laboratory} ($\azdblab$), to assist database researchers 
in conducting a {\em large-scale} empirical study across multiple \hbox{DBMSes} in a {\em scientific} manner. 
For them to test their hypotheses on the behavior of query \hbox{optimizers}, 
$\azdblab$ can run and monitor a large-scale \hbox{experiment} with thousands (or millions) of queries on 
\hbox{different} \hbox{DBMSes}. 
Furthermore, $\azdblab$ can help users \hbox{automatically} analyze these queries. 
We elaborate in detail how $\azdblab$ seamlessly supports scientific research on a variety of relational \hbox{DBMSes}.
\end{abstract}

\keywords{}

\maketitle

\section{Introduction\label{sec:introduction}}

\section{Background}\label{sec:background}

\subsection{Accuracy and Precision in Timing}\label{sec:prec_accuracy}

\section{Conclusion}
In this paper, we presented a novel laboratory information system 
for empirical studies of different relational DBMSes.

\section{Future Work}\label{sec:future_work}

\section{Acknowledgments}
We give special thanks to Rui Zhang, for his initial development on $\azdblab$. 
We also thank Ricardo Carlos, Preetha Chatterjee, Pallavi
Chilappagari, Jennifer Dempsey, David Gallup, Kevan Holdaway, Andrey
Kvochko, Siou Lin, Adam Robertson, Lopamudra Sarangi, Linh Tran, Cheng Yi, and Man Zhang for
their contributions to the $\azdblab$. and Phil Kaslo, Tom Lowry, and John
Luiten for constructing and maintaining our experimental instrument\c2j{}{,
  a laboratory of six machines and associated software}. 
This research was supported in part by NSF grants \hbox{IIS-0639106} and \hbox{IIS-1016205}.

\bibliographystyle{wileyj}
%\bibliography{paper}

\end{document}

