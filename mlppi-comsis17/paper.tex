\documentclass[runningheads]{comsis2}

%\usepackage[ruled,linesnumbered]{algorithm2e}
%\usepackage{amssymb,amsthm,amsmath}
\usepackage{balance}
\usepackage{subfigure}
\usepackage{float}
%\newtheorem{theorem}{Theorem}[section]
%\newtheorem{lemma}[theorem]{Lemma}
\usepackage{latexsym}
\usepackage{multirow}
\usepackage{color}
\usepackage{url}
\usepackage{comment}
\usepackage[dvips]{graphicx}          % graphicx package for including ps files  

%% Necessary definitions for the running heads
\def\journalissue{Computer Science and Information Systems ?(?):??--??}
\def\paperidnum{DOI: N/A}
\setcounter{page}{1}

\title{An Automated Multi-level Partitioning Tool on Analytical Workloads}
%\titlenote{An earlier version of this article was presented as a short paper at ACM CIKM'12.}
%% Use this if the title is too long for the running heads
%\titlerunning{An Automated Multi-level Partitioning Tool on Analytical Workloads}

\author{Young-Kyoon Suh\inst{1} \and Alain Crolotte\inst{2} \and Pekka Kostamaa\inst{2}}
%Correspondence to Young-Kyoon Suh, now at Korea Institute of Science and Technology Information
%% Use this the list of authors is too long for the running heads
%\authorrunning{Correspondence to Young-Kyoon Suh, now at Korea Institute of Science and Technology Information}

\institute{Department of Computer Science, The University of Arizona, Tucson, AZ 85721, USA\\ 
  \email{yksuh@cs.arizona.edu}
  \and
  Teradata Corporation, El Segundo, CA 90245, USA\\
  \email{\{alain.crolotte,pekka.kostamaa\}@teradata.com}
%  \and
%  Korea Institute of Science and Technology Information, Daejeon 34141 Republic of Korea\\ 
%  \email{yksuh@kisti.re.kr}
}

\def\form#1{$\langle{#1}\rangle$}
\def\range#1{$[{#1}]$}
\def\openrange#1{$({#1})$}
\def\lopenrange#1{$({#1}]$}
\def\ropenrange#1{$[{#1})$}
\def\product#1{$(#1)$}
\definecolor{grey}{RGB}{200,200,200}
\newcommand{\hilite}[1]{\colorbox{grey}{#1}}
\newcommand{\hilitey}[1]{\colorbox{yellow}{#1}}
\newcommand{\hiliting}[1]{\colorbox{grey}{#1}}
\long\def\todo#1{\hilitey{{\bf TODO:} {\em #1}}}
\long\def\shorten#1{}
\long\def\comment#1{}
\def\MAX{\mbox{\sl MAX}}
\def\LIMIT{\mbox{\sl LIMIT}}
\def\getQueryCost{\mbox{\sl getQueryCost}}
\def\getScanCost{\mbox{\sl getScanCost}}
\def\BigO{\mbox{\sl O}}
\def\NULL{\mbox{\sl NULL}}
\def\SolA{\mbox{\sl A}}
\def\SolB{\mbox{\sl B}}
\def\SolC{\mbox{\sl C}}
\def\SolS{\mbox{\sl S}}
\def\Qone{\mbox{\sl q1}}
\def\Qtwo{\mbox{\sl q2}}
\def\Sone{\mbox{\sl s1}}
\def\Stwo{\mbox{\sl s2}}
\def\SolSone{\mbox{\sl S1}}
\def\SolStwo{\mbox{\sl S2}}
\def\SolSthree{\mbox{\sl S3}}
\def\SolSfour{\mbox{\sl S4}}
\def\SolSfive{\mbox{\sl S5}}

\begin{document}

\maketitle

\begin{abstract} 
Typically, it is a daunting task for a database administrator (DBA) to devise a solution to partitioning a huge fact table accessed by query workloads, for better performance. To relieve such a burden for the DBA, we introduce an intelligent physical database designer to recommend an optimized partitioning on the fact table. This designer uses a greedy algorithm for search space enumeration. This space is driven by predicates of a given query workload. The designer takes advantage of the cost model of a query optimizer to prune the search space. The wizard resides completely on a client and interacts with the optimizer via APIs. Thus, there is no overhead to instrument the optimizer code. Furthermore, the predicate-driven method can be applied to any clustering or partitioning schemes. We show that the designer’s recommendation outperforms a human expert’s solution. We also demonstrate that the recommendation scale very well with increasing workload and growing fact table.

\vspace{6pt}\textbf{Keywords:} keyword1, keyword2.
\end{abstract}

\section{Introduction}

The text of the paper\dots


%\bibliographystyle{splncs03}
%\bibliography{example}


%%%%%%%%%%%%%%%%%%%%%%%%%%%%%%%%%%%%%%%%%
%% For the final version of the paper: %%
%%%%%%%%%%%%%%%%%%%%%%%%%%%%%%%%%%%%%%%%%

%% Author information
%\vspace{4ex}\noindent
%\textbf{Author One} is\dots
%
%\bigskip\noindent
%\textbf{Author Two} is\dots
%
%\bigskip\noindent
%\textbf{Author Three} is\dots

%% Reception and acceptance information
%\bigskip
%\paragraph{Received: Month DD, 20YY; Accepted: Month DD, 20YY.}

\end{document}
