\documentclass[10pt,letterpaper]{article}

\usepackage{graphicx}
\usepackage{longtable}
\usepackage{algorithmic}
\usepackage{alltt}
\usepackage{array}
\usepackage[cmex10]{amsmath}
\usepackage{amssymb}
%\usepackage[caption=false]{caption}
\usepackage[dvipsnames]{color}
%\usepackage{colortbl}
%\usepackage{enumitem}
%\usepackage{eqparbox}
%\usepackage{fixltx2e}
%\usepackage{float}
%\usepackage{floatflt}
%\usepackage{mdwmath}
%\usepackage{mdwtab}
%\usepackage{multirow}
\usepackage{stfloats}
%\usepackage[font=footnotesize]{subfig}
\usepackage[caption=false,font=normalsize,labelfont=sf,textfont=sf]{subfig}
%\usepackage[tight,normalsize,sf,SF]{subfigure}
%\usepackage{subfigure} 
\usepackage{times}
\usepackage{url}
\usepackage{verbatim} 
\usepackage{wrapfig}
\usepackage{xspace}
\usepackage{color}

\definecolor{grey}{RGB}{200,200,200}
\newcommand{\hilite}[1]{\colorbox{yellow}{#1}}
\newcommand{\hiliting}[1]{\colorbox{grey}{#1}}
\long\def\todo#1{\hilite{{\bf TODO:} {\em #1}}}

\setlength{\oddsidemargin}{0in}
\setlength{\textwidth}{6.5in}


% new commands: from file -- to permit re-use
%\newcommand{\ie}{\textit{i.e.},\xspace}
\newcommand{\eg}{\textit{e.g.},\xspace}
\newcommand{\cf}{\textit{cf.},\xspace}
\newcommand{\etc}{\textit{etc}.\@\xspace}
\newcommand{\vs}{\textit{vs}}
%\newcommand{\tVal}{{\sc $\tau$Validator}}
\newcommand{\tVal}{{\sc $\tau$XMLLint}}
\newcommand{\tX}{$\tau$XSchema}
\def\tB{\mbox{$\tau$Bench}}
\newcommand{\TODO}[1]{[TODO: #1]} 
\newcommand{\DONE}[1]{[DONE: #1]} 
\newcommand{\note}[1]{\colorbox{yellow}{\emph{#1}}}
\newcommand{\stevecomment}[1]{}
\newcommand{\tab}{\hspace{5mm}}



%%%%%%%%%%%%%%%%%%%%%%%%%%%%%%%%%%%%%%%%%%%%%%%%%%
%%% Commands taken from Steve's Thesis %%%%%%%%%%%%%%%%%%%%%%%%%%
\newcommand{\tool}[1]{{\sc #1}}
%\newcommand{\ex}[1]{ \hbox{\tt\small<#1>}} 
%\newcommand{\att}[1]{ \hbox{\tt\small#1}} 
\newcommand{\ele}[1]{\hbox{\tt <#1>}} 
\newcommand{\att}[1]{\hbox{\tt #1}} 
\newcommand{\attval}[1]{{\tt "#1"}} 
\newcommand{\tXml}{{\sc $\tau$XMLLint}}
\newcommand{\xmllint}{{\sc XMLLint}}
\newcommand{\squash}{{\sc Squash}}
\newcommand{\unsquash}{{\sc UnSquash}}
\newcommand{\resquash}{{\sc ReSquash}}
\newcommand{\validator}{{\em Temporal Constraint Validator Module}}
\newcommand{\schemamapper}{{\sc SchemaMapper}}
\newcommand{\putFile}[1]{\lstset{caption={\tt #1},label=listing:#1} \lstinputlisting{examples/#1}}
\newcommand{\putFileNL}[1]{\lstset{caption={\tt #1},label=listing:#1,numbers=none} \lstinputlisting{examples/#1} \lstset{label=,numbers=left}}
\newcommand{\putFileNLCompany}[1]{\lstset{language=CompanyXML} \lstset{caption={\tt #1},label=listing:#1,numbers=none} \lstinputlisting{examples/#1} \lstset{label=,numbers=left}\lstset{language=SteveXML} }
\newcommand{\putFileRep}[1]{\lstset{caption={\tt #1},label=listing:#1} \lstinputlisting{examples/exampleRep/#1}}
\newcommand{\putFileRepCapNL}[2]{\lstset{caption=#2,label=listing:#1,numbers=none} \lstinputlisting{examples/#1} \lstset{label=,numbers=left}}
\newcommand{\putFileRepCap}[2]{\lstset{caption=#2,label=listing:#1} \lstinputlisting{examples/exampleRep/#1} \lstset{label=}}

\newcommand{\needcite}{\colorbox{yellow}{\textbf{[?]}}}

% Counters for enumerate - makes bold letters (eg, (a)) and bold first few words.
\newcommand{\litem}[1]{\item{\bfseries #1}}
\newcounter{enumi_saved}
\newcommand{\startlist}{\begin{enumerate}[label=\textbf{(\alph{*})}]  \setcounter{enumi}{\value{enumi_saved}}}
\newcommand{\listdone}{\setcounter{enumi_saved}{\value{enumi}}\end{enumerate}}

% Counters for enumerate - makes bold letters (eg, (a)) and bold first few words.
\newcounter{denumi_saved}
\newcommand{\startlistD}{\begin{enumerate}[label=\textbf{(\arabic{*})}]\setcounter{enumi}{\value{denumi_saved}}}
\newcommand{\listdoneD}{\setcounter{denumi_saved}{\value{enumi}}\end{enumerate}}

% Smaller lists (bullets)
\newcommand{\squishlist}{
   \begin{list}{$\bullet$}
    { \setlength{\itemsep}{0pt}      \setlength{\parsep}{1pt}
      \setlength{\topsep}{1pt}       \setlength{\partopsep}{0pt}
      \setlength{\leftmargin}{2.8em} \setlength{\labelwidth}{1em}
      \setlength{\labelsep}{0.5em} } }
      
% Smaller lists (dash)
\newcommand{\squishdash}{
   \begin{list}{-}
    { \setlength{\itemsep}{0pt}      \setlength{\parsep}{1pt}
      \setlength{\topsep}{1pt}       \setlength{\partopsep}{0pt}
      \setlength{\leftmargin}{2.8em} \setlength{\labelwidth}{1em}
      \setlength{\labelsep}{0.5em} } }
\newcommand{\squishend}{
    \end{list}  }

% Smaller lists (numbers)
\newcommand{\numsquishlist}{
   \begin{list}{$\bullet$}
    { \setlength{\itemsep}{0pt}      \setlength{\parsep}{1pt}
      \setlength{\topsep}{1pt}       \setlength{\partopsep}{0pt}
      \setlength{\leftmargin}{2.8em} \setlength{\labelwidth}{1em}
      \setlength{\labelsep}{0.5em} } }

\newcommand{\numsquishend}{
    \end{list}  }


%Hyphenation problems
\hyphenation{name-space}
\hyphenation{schema-location}
\hyphenation{xml-lint}
\hyphenation{time-stamp}
\hyphenation{time-stamps}


%Hyphenation problems
\hyphenation{name-space}
\hyphenation{schema-location}
\hyphenation{xml-lint}
\hyphenation{time-stamp}
\hyphenation{time-stamps}
\hyphenation{time-stamped}


\urlstyle{rm}
\graphicspath{{./figures/}}

\definecolor{gray}{rgb}{0.5,0.5,0.5}
%\newcommand{\rev}[1]{\textcolor{blue}{#1}}
%\newcommand{\quo}[1]{\textcolor{gray}{#1}}
%\newcommand{\rev}[1]{\vspace{3mm} \noindent {\em {\color{blue}{#1}}}}


\newenvironment{myindentpar}[1]%
{\begin{list}{}
         {\vspace{10pt}
					\setlength{\leftmargin}{#1}}
          \item[]
}
{\end{list}}
\newcommand{\rev}[1]{\begin{myindentpar}{.25in} {\em {\color{blue}{#1}}}\end{myindentpar}}

\newenvironment{myindentparQ}[1]%
{\begin{list}{}
         {\setlength{\leftmargin}{#1}}
          \item[]
}
{\end{list}}
\newcommand{\quo}[1]{\begin{myindentparQ}{.25in} {{\color{gray}{#1}}}\end{myindentparQ}}

\begin{document}

\title{Author's Reply}
\author{}
\maketitle

\section*{Overview}\label{sec:overview}
I appreciate these quite thorough reviews. I appreciate the feedback and suggestions.

\newcounter{RP}

\clearpage
\section*{Reviewer}\label{sec:rev1}

% 1
\rev{
$<<$ Reviewer's comments to the author(s) $>>$

The SEDONA is based on the probability that any daemons run during the
measurement. And cutoff time is based on the actual measurement time.
The proposed approach is interesting, but I cannot understand why the
author employs such statistics and heuristic way, and how efficient the
proposed approach will work. Please refer to my comments, and consider
to revise the manuscript and re-submit .
}

I understand your concern. I respond to individually below. 
Please also note that your (excellent) comments are incorporated into the revision. 

\rev{
Comments: 
(1) In the last of the manuscript, the author describes ``Our plan is to
integrate SEDONA into the query timing protocol [8]." If the research
goal is the combination of the SEDONA and the query timing protocol,
the proposed approach may be suitable. However, the goal is not
clearly described in the manuscript. The readers will recognize the
SEDONA is a generic protocol. If the research goal is the combination
of this work and the prior work, the author should mention so.
}

I understand that the goal of this paper is not well articulated. 
Yes, SEDONA is a generic protocol, which can be applied to any arbitrary Linux system 
when it comes to reducing variability of the execution time of a program of interest. 
I now state the actual goal, marked in blue, in the introduction section. 

Note also that the mention of my plan in the future work 
is just one aspect of the extension of the SEDONA protocol. 
I claim that the combination is not the ultimate research goal of this paper. 
To avoid confusion, I removed that sentence. 

\rev{
(2) If the proposed approach is for a generic purpose, there is a strong
doubt about usability. The improvement of the SEDONA is little. And the
extent of the each benchmark’s improvement described in Table 3
varies widely. Thus, the experimental results does not confirm whether
the SEDONA is widely practical or is useful. The description to clarify
the SEDONA’s characteristics is needed.
} 

\rev{ 
(3) The improvement of the SEDONA is little. And the extent of the each
benchmark’s improvement described in Table 3 varies widely. Thus, the
experimental results does not confirm whether the SEDONA is widely
practical or is useful.
}

This review is the same as (2). Please refer to our response of (2) for this comment.

\rev{ 
(4) Since the explanation of ``dual-PUT'' is not enough, I can not well
understand the algorithm shown in Fig. 3. Similarly, what is “1st
Half’s ET (ms)” attached at Fig. 4(b), 4(c), 4(d) ? I understood only that
the label is related with “dual-PUT”. These words are important in the
manuscript. So, please keep in mind to write understandably.
}

\end{document}
