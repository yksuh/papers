\documentclass[10pt,letterpaper]{article}

\usepackage{graphicx}
\usepackage{longtable}
\usepackage{algorithmic}
\usepackage{alltt}
\usepackage{array}
\usepackage[cmex10]{amsmath}
\usepackage{amssymb}
%\usepackage[caption=false]{caption}
\usepackage[dvipsnames]{color}
%\usepackage{colortbl}
%\usepackage{enumitem}
%\usepackage{eqparbox}
%\usepackage{fixltx2e}
%\usepackage{float}
%\usepackage{floatflt}
%\usepackage{mdwmath}
%\usepackage{mdwtab}
%\usepackage{multirow}
\usepackage{stfloats}
%\usepackage[font=footnotesize]{subfig}
\usepackage[caption=false,font=normalsize,labelfont=sf,textfont=sf]{subfig}
%\usepackage[tight,normalsize,sf,SF]{subfigure}
%\usepackage{subfigure} 
\usepackage{times}
\usepackage{url}
\usepackage{verbatim} 
\usepackage{wrapfig}
\usepackage{xspace}
\usepackage{color}

\definecolor{grey}{RGB}{200,200,200}
\newcommand{\hilite}[1]{\colorbox{yellow}{#1}}
\newcommand{\hiliting}[1]{\colorbox{grey}{#1}}
\long\def\todo#1{\hilite{{\bf TODO:} {\em #1}}}

\setlength{\oddsidemargin}{0in}
\setlength{\textwidth}{6.5in}


% new commands: from file -- to permit re-use
%\newcommand{\ie}{\textit{i.e.},\xspace}
\newcommand{\eg}{\textit{e.g.},\xspace}
\newcommand{\cf}{\textit{cf.},\xspace}
\newcommand{\etc}{\textit{etc}.\@\xspace}
\newcommand{\vs}{\textit{vs}}
%\newcommand{\tVal}{{\sc $\tau$Validator}}
\newcommand{\tVal}{{\sc $\tau$XMLLint}}
\newcommand{\tX}{$\tau$XSchema}
\def\tB{\mbox{$\tau$Bench}}
\newcommand{\TODO}[1]{[TODO: #1]} 
\newcommand{\DONE}[1]{[DONE: #1]} 
\newcommand{\note}[1]{\colorbox{yellow}{\emph{#1}}}
\newcommand{\stevecomment}[1]{}
\newcommand{\tab}{\hspace{5mm}}



%%%%%%%%%%%%%%%%%%%%%%%%%%%%%%%%%%%%%%%%%%%%%%%%%%
%%% Commands taken from Steve's Thesis %%%%%%%%%%%%%%%%%%%%%%%%%%
\newcommand{\tool}[1]{{\sc #1}}
%\newcommand{\ex}[1]{ \hbox{\tt\small<#1>}} 
%\newcommand{\att}[1]{ \hbox{\tt\small#1}} 
\newcommand{\ele}[1]{\hbox{\tt <#1>}} 
\newcommand{\att}[1]{\hbox{\tt #1}} 
\newcommand{\attval}[1]{{\tt "#1"}} 
\newcommand{\tXml}{{\sc $\tau$XMLLint}}
\newcommand{\xmllint}{{\sc XMLLint}}
\newcommand{\squash}{{\sc Squash}}
\newcommand{\unsquash}{{\sc UnSquash}}
\newcommand{\resquash}{{\sc ReSquash}}
\newcommand{\validator}{{\em Temporal Constraint Validator Module}}
\newcommand{\schemamapper}{{\sc SchemaMapper}}
\newcommand{\putFile}[1]{\lstset{caption={\tt #1},label=listing:#1} \lstinputlisting{examples/#1}}
\newcommand{\putFileNL}[1]{\lstset{caption={\tt #1},label=listing:#1,numbers=none} \lstinputlisting{examples/#1} \lstset{label=,numbers=left}}
\newcommand{\putFileNLCompany}[1]{\lstset{language=CompanyXML} \lstset{caption={\tt #1},label=listing:#1,numbers=none} \lstinputlisting{examples/#1} \lstset{label=,numbers=left}\lstset{language=SteveXML} }
\newcommand{\putFileRep}[1]{\lstset{caption={\tt #1},label=listing:#1} \lstinputlisting{examples/exampleRep/#1}}
\newcommand{\putFileRepCapNL}[2]{\lstset{caption=#2,label=listing:#1,numbers=none} \lstinputlisting{examples/#1} \lstset{label=,numbers=left}}
\newcommand{\putFileRepCap}[2]{\lstset{caption=#2,label=listing:#1} \lstinputlisting{examples/exampleRep/#1} \lstset{label=}}

\newcommand{\needcite}{\colorbox{yellow}{\textbf{[?]}}}

% Counters for enumerate - makes bold letters (eg, (a)) and bold first few words.
\newcommand{\litem}[1]{\item{\bfseries #1}}
\newcounter{enumi_saved}
\newcommand{\startlist}{\begin{enumerate}[label=\textbf{(\alph{*})}]  \setcounter{enumi}{\value{enumi_saved}}}
\newcommand{\listdone}{\setcounter{enumi_saved}{\value{enumi}}\end{enumerate}}

% Counters for enumerate - makes bold letters (eg, (a)) and bold first few words.
\newcounter{denumi_saved}
\newcommand{\startlistD}{\begin{enumerate}[label=\textbf{(\arabic{*})}]\setcounter{enumi}{\value{denumi_saved}}}
\newcommand{\listdoneD}{\setcounter{denumi_saved}{\value{enumi}}\end{enumerate}}

% Smaller lists (bullets)
\newcommand{\squishlist}{
   \begin{list}{$\bullet$}
    { \setlength{\itemsep}{0pt}      \setlength{\parsep}{1pt}
      \setlength{\topsep}{1pt}       \setlength{\partopsep}{0pt}
      \setlength{\leftmargin}{2.8em} \setlength{\labelwidth}{1em}
      \setlength{\labelsep}{0.5em} } }
      
% Smaller lists (dash)
\newcommand{\squishdash}{
   \begin{list}{-}
    { \setlength{\itemsep}{0pt}      \setlength{\parsep}{1pt}
      \setlength{\topsep}{1pt}       \setlength{\partopsep}{0pt}
      \setlength{\leftmargin}{2.8em} \setlength{\labelwidth}{1em}
      \setlength{\labelsep}{0.5em} } }
\newcommand{\squishend}{
    \end{list}  }

% Smaller lists (numbers)
\newcommand{\numsquishlist}{
   \begin{list}{$\bullet$}
    { \setlength{\itemsep}{0pt}      \setlength{\parsep}{1pt}
      \setlength{\topsep}{1pt}       \setlength{\partopsep}{0pt}
      \setlength{\leftmargin}{2.8em} \setlength{\labelwidth}{1em}
      \setlength{\labelsep}{0.5em} } }

\newcommand{\numsquishend}{
    \end{list}  }


%Hyphenation problems
\hyphenation{name-space}
\hyphenation{schema-location}
\hyphenation{xml-lint}
\hyphenation{time-stamp}
\hyphenation{time-stamps}


%Hyphenation problems
\hyphenation{name-space}
\hyphenation{schema-location}
\hyphenation{xml-lint}
\hyphenation{time-stamp}
\hyphenation{time-stamps}
\hyphenation{time-stamped}


\urlstyle{rm}
\graphicspath{{./figures/}}

\definecolor{gray}{rgb}{0.5,0.5,0.5}
%\newcommand{\rev}[1]{\textcolor{blue}{#1}}
%\newcommand{\quo}[1]{\textcolor{gray}{#1}}
%\newcommand{\rev}[1]{\vspace{3mm} \noindent {\em {\color{blue}{#1}}}}


\newenvironment{myindentpar}[1]%
{\begin{list}{}
         {\vspace{10pt}
					\setlength{\leftmargin}{#1}}
          \item[]
}
{\end{list}}
\newcommand{\rev}[1]{\begin{myindentpar}{.25in} {\em {\color{blue}{#1}}}\end{myindentpar}}

\newenvironment{myindentparQ}[1]%
{\begin{list}{}
         {\setlength{\leftmargin}{#1}}
          \item[]
}
{\end{list}}
\newcommand{\quo}[1]{\begin{myindentparQ}{.25in} {{\color{gray}{#1}}}\end{myindentparQ}}

\begin{document}

\title{Author's Reply}
\author{}
\maketitle

\section*{Overview}\label{sec:overview}
I appreciate these quite thorough reviews. I appreciate the feedback and suggestions.

\newcounter{RP}

\clearpage
\section*{Reviewer}\label{sec:rev1}

% 1
\rev{
$<<$ Reviewer's comments to the author(s) $>>$

The SEDONA is based on the probability that any daemons run during the
measurement. And cutoff time is based on the actual measurement time.
The proposed approach is interesting, but I cannot understand why the
author employs such statistics and heuristic way, and how efficient the
proposed approach will work. Please refer to my comments, and consider
to revise the manuscript and re-submit .
}

Your concerns are well understood. I found such statistics and heuristic approaches 
very useful to catch what system daemons are infrequent, long-running. 
Also, the approaches make our protocol generic, in that 
such statistics and heuristics can be computed and 
applied on any Linux system. The experimental results show that 
our protocol is more efficient than the traditional method relying on elapsed time. 

I now respond to individually below.
%Your comments are incorporated into the revision. 

\rev{
Comments: 
(1) In the last of the manuscript, the author describes ``Our plan is to
integrate SEDONA into the query timing protocol [8]." If the research
goal is the combination of the SEDONA and the query timing protocol,
the proposed approach may be suitable. However, the goal is not
clearly described in the manuscript. The readers will recognize the
SEDONA is a generic protocol. If the research goal is the combination
of this work and the prior work, the author should mention so.
}

I understand your confusion. The previously described plan was suggested as one of the possibilities in the future work. But that is not the research goal of this paper. 
As you already recognized, the goal of this paper is to propose a {\em generic, better} timing protocol against the conventional approach of utilizing elapsed time, 
to enable better timing results. This goal is successfully achieved 
by significantly reducing variability in timing as demonstrated in our various experiments.
%The goal is successfully achieved by significantly reducing variability of 
%the execution time of a program of interest in our experiments. 
I now clearly state this goal, marked in red, at the last sentence in Section~1. 

To avoid confusion, I removed the last mention. Also, the first paragraph on page~2, 
which used to provide an explanation of a possible combination of the query timing protocol 
with SEDONA, was removed.

%Note also that the mention of my plan in the future work 
%is just one aspect of the extension of the SEDONA protocol. 
%I claim that the combination is not the ultimate research goal of this paper. 
%To avoid confusion, I removed that sentence. 

\rev{
(2) If the proposed approach is for a generic purpose, there is a strong
doubt about usability. The improvement of the SEDONA is little. And the
extent of the each benchmark’s improvement described in Table 3
varies widely. Thus, the experimental results does not confirm whether
the SEDONA is widely practical or is useful. The description to clarify
the SEDONA’s characteristics is needed.
} 

Your concern is understood, but it is hard to agree with the comment that the improvement is little, for the following reasons.
First, for a certain workload such as {\tt 434}, 
our protocol reduced the variability was reduced up to by about 10x (from 75 ms to 7 ms), 
which is *not* small. (Please refer to the last sentence 
in the first paragraph in the second column on page~4.)
There were also some other workloads such as {\tt 410} and {\tt 445} 
in which the improvement reached about 3x, respectively. 
Furthermore, every benchmark's standard deviation by our protocol 
was less than that of the conventional method. 
The relative error of the benchmarks was on average improved 
by 1.5x as well.

Second, our protocol scaled well over increasing execution time. 
(Please refer to the second paragraph in the right column on page~4.)
The relative error of our protocol was still smaller than that of the ORG method 
for a long-running benchmark like {\tt 436} and {\tt 454} 
as well as for a medium-length or a short-length one. 
Such scalability was still observed in our new experiment, 
which I describe shortly, as shown in Fig.~5.

Note also that the ET results by the ORG method were also based 
on the settings taking care of several identified timing factors as
described in the third paragraph in Section~2.
Under such a well-configured timing environment, our protocol further 
reduced the variability of execution time compared to the ORG method
in such various benchmarks, 
which actually demonstrated its general purpose and wide usability. 

Lastly, the results from a new experiment provide 
another strongly evidence that our protocol is widely practical and useful. 
In the new experiment we evaluated the performance of our protocol 
on some real-world programs like insertion sort and matrix multiplication (in column major). 
The experimental results showed that the performance gap between the original and our methods 
was increasingly widened, e.g., up to by 6x, as workload level increased. 

Therefore, I strongly claim that these various experiments confirm 
the wide acceptance and usability of SEDONA as an alternative to the ET-based measurement technique. 

\rev{ 
(3) The improvement of the SEDONA is little. And the extent of the each
benchmark’s improvement described in Table 3 varies widely. Thus, the
experimental results does not confirm whether the SEDONA is widely
practical or is useful.
}

This review is basically the same as (2). Please refer to our response of (2) for this comment. 
One note. You may feel that the extent of the improvement varies widely, 
because the SPEC CPU2006 includes a set of very different compute-bound benchmarks. 
But even if so, we found that all the benchmarks in the suite benefited from our timing protocol, 
and in addition, for some workloads, our protocol significantly reduced their variability up to by 10x.  


\rev{ 
(4) Since the explanation of ``dual-PUT'' is not enough, I can not well
understand the algorithm shown in Fig. 3. Similarly, what is “1st
Half’s ET (ms)” attached at Fig. 4(b), 4(c), 4(d) ? I understood only that
the label is related with “dual-PUT”. These words are important in the
manuscript. So, please keep in mind to write understandably.
}

Your point is well taken. 
The term of ``dual-PUT'' represents 
a pair of two consecutive samples of a run of a (regular) PUT. 
(The length of that regular PUT is virtually doubled as dual-PUT.) More descriptions of the term ``\hbox{dual-PUT}'' are provided with an easy example 
in the second paragraph in the right column on page~2. 
In addition, the 1st half indicates the first element of the pair 
whereas the 2nd half represents the second element of the same pair. 
The label of ``1st Half's ET (ms)'' is renamed as ``Odd Samples' ET(ms),'' 
for better clarification. Please refer to the following (third) paragraph. 


\end{document}
